\begin{longtable}[]{@{}
  >{\raggedright\arraybackslash}p{(\columnwidth - 4\tabcolsep) * \real{0.2435}}
  >{\raggedright\arraybackslash}p{(\columnwidth - 4\tabcolsep) * \real{0.3757}}
  >{\raggedright\arraybackslash}p{(\columnwidth - 4\tabcolsep) * \real{0.3808}}@{}}
\toprule()
\begin{minipage}[b]{\linewidth}\raggedright
Objectives
\end{minipage} & \begin{minipage}[b]{\linewidth}\raggedright
Requirements/\emph{Evaluation Criteria}
\end{minipage} & \begin{minipage}[b]{\linewidth}\raggedright
Justification
\end{minipage} \\
\midrule()
\endhead
\multicolumn{3}{@{}>{\raggedright\arraybackslash}p{(\columnwidth - 4\tabcolsep) * \real{1.0000} + 4\tabcolsep}@{}}{%
\textbf{Goal 1: The device corrects the users posture}} \\
1. Shall encourage the user to sit without hunching or rounded
shoulders.

1.5 Shall only allow poor posture for some maximum recommended time &
The user shall sit with a symmetrical trunk posture, with a trunk
inclination between 100-110 degrees {[}24{]}, and spine posture as
specified by ISO 11226:2000 {[}25{]}.

\emph{The more time we can comfortably sit in this position, the
better.}

\emph{Comfortability will be measured according to the pain and
tiredness of the muscles -\/-\textgreater{} Discomfort will be
considered as soon as one of these two arrive -\/-\/-\textgreater{} USER
TESTING}

The user shall be allowed to sit with poor posture for a \emph{maximum
holding time} recommended by ISO 11226:2000, depending on the user's
trunk inclination {[}25{]}. & Posture research has informed ISO
11226:2000 {[}25{]} of optimal seating position for good posture. The
user ideally follows this standard. \\
& & \\
\multicolumn{3}{@{}>{\raggedright\arraybackslash}p{(\columnwidth - 4\tabcolsep) * \real{1.0000} + 4\tabcolsep}@{}}{%
\textbf{Goal 2: The device is safe to use}} \\
1. Materials used shall have the least amount possible of carcinogens
and allergens & The number of carcinogens and allergens shall conform to
the maximum carcinogens and allergens imposed by the OEKO-TEX Standard
100 {[}12{]}.

The product should not contain flame retardant (FR) chemicals more than
1000 parts per million (ppm).

\emph{The fewer maximum carcinogens and allergens imposed the better.} &
Research has been done to maximize potential users and be the least
harmful possible.

As required by
\href{https://bhgs.dca.ca.gov/forms_pubs/ab2998_faq.pdf}{AB 2998} is a
bill enforced by the Bureau of Household Goods and Services (BHGS) in
California. However, this is a good requirement to ensure safety and it
is usable in different areas and does not interfere with the materials
inside the pillow (as metals and fabrics can easily catch fire).
{[}32{]} \\
2. Shall not be an electrical hazard for the user & The circuits in the
materials shall conform to the maximum voltage dictated by UL 60601-1
{[}13{]}.

\emph{The less voltage required, the better.}

Shall conform to Health Canada standard for Limits of Human Exposure to
Radiofrequency Electromagnetic Energy in the Frequency Range from 3 kHz
to 300 GHz & \\
3. Shall not provide a lot of pressure & Max uncomfortable is
20-40g/cm\^{}2 & \\
4. Easy way to get out of it & Shall conform to the safe distances for
different body parts specified in ISO 10535 to avoid body traps. A table
of distances for different body parts is included in the Source Extracts
& If the user is a safe distance away from mechanisms, they can leave
the product quickly in case of an emergency \\
5. The product should be comfortable to sit on or be in contact with for
a prolong period of time. & The product has a Kawabata Evaluation Score
of 4 and higher, measured and tested using the Kawabata Evaluation
System. & From {[}35{]} and {[}36{]}, the Kawabata tool is widely used
in fabrics and materials, and a score of 4 to 5 is good and excellent,
which we want to achieve to maintain high comfort, as this was a need
presented by our stakeholders. \\
\multicolumn{3}{@{}>{\raggedright\arraybackslash}p{(\columnwidth - 4\tabcolsep) * \real{1.0000} + 4\tabcolsep}@{}}{%
\textbf{Goal 3: The device is durable}} \\
1. Shall not break during normal operation & Shall be operational when
placed in 3K21 conditions (temperature controlled from 15°C-32°C, but
not humidity controlled as specified by IEC 60721-3-3 {[}14{]}

\emph{The longer the product remains completely operational, the
better.} & EngSci students study in many places so the product should
withstand daily indoor wear and tear. \\
2. Shall not be damaged by household cleaning supplies & Shall maintain
the same mass, dimensions, and appearance after drying from being
immersed in household cleaning supplies (specified in Annex A of ISO
175. Examples include acetic acid, ethanol, and hydrogen peroxide).
{[}15{]}

\emph{The longer the product remains immersed in a cleaning fluid
without changing mass, size, or appearance, the better} & EngSci
students will clean the product~so it will not be damaged by cleaning
supplies. We can also extrapolate these results to conclude that body
oils and sweat will not damage the product because cleaning supplies are
much more basic or alkaline than human sweat/oils {[}16{]}. \\
\multicolumn{3}{@{}>{\raggedright\arraybackslash}p{(\columnwidth - 4\tabcolsep) * \real{1.0000} + 4\tabcolsep}@{}}{%
\textbf{Goal 4: The device is portable}} \\
1. Shall have small dimensions. & Shall have dimensions no bigger than 7
x 7 x 7 inches when in travel mode (i.e. when being carried and not in
use)

Shall be able to hold with one hand.

\emph{The smaller and more~compact the product, the better.} & The ideal
phone size is 6.1 inches, as it is portable enough to fit anywhere you
go but it is not too small to function {[}17{]}. Based off this, to
construct something that can be taken everywhere, almost like a phone,
it is fair if it is in similar dimensions. \\
2. Shall be lightweight & The product shall weigh no more than 600g
(based of 500g from reference designs).

\emph{The lower the mass, the better.} & The original back pod design
(pg 7) has a mass of 500g {[}10{]}, and the ideal phone has a mass of
130g {[}21{]} justifying this mass. \\
3.~Skin-contacting material should be breathable and should not impede
the movement of the individual. & Material should be made of soft
silicone, memory foam, cotton/elastic blends, canvas, neoprene or
anything with a Thermal Evaporative Resistance (RET) coefficient
\textless{} 6 (tested using ISO 11092 standard) {[}18{]}{[}19{]}. If
fabric is used, it should be composed of nylon, elastic or materials
sharing similar properties {[}20{]}.

\emph{A lower} \emph{(RET) Coefficient is better} & Lower RET values
mean more breathable materials that will keep users comfortable
throughout the day.

Cotton/elastic blends and similar materials are flexible, allowing for
less pressure applied on the back, making it easier to wear the
material. \\
\multicolumn{3}{@{}>{\raggedright\arraybackslash}p{(\columnwidth - 4\tabcolsep) * \real{1.0000} + 4\tabcolsep}@{}}{%
\textbf{Goal 5: The device is aesthetically pleasing}} \\
1. Shall not be bulky or any weird shapes & The product shall not have
sharp corners or radii {[}22{]}.

\emph{The larger the radii of curvature at corners, the better} & Sharp
corners are both dangerous and uncomfortable to have in contact with
skin. \\
2. Hardware shall not be visible or distracting & The colour of the
product shall not be any neon colour.

\emph{The lower the visibility of the product while in use in public,
the better.} & From {[}23{]}, users of back braces did not wear them for
the prescribed times as they thought the hardware being visible had an
affect on their confidence. \\
\bottomrule()
\end{longtable}
