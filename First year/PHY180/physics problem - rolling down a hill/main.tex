\documentclass[12pt]{article}
\usepackage[tmargin=1in,bmargin=1in,lmargin=1in,rmargin=1in]{geometry}
\usepackage{mypackagesv2}
\setlength\parindent{0pt}

\begin{document}


\textbf{An alternate approach:} consider the angular momentum measured from the center of mass and the net force of the ball. \\

Set up forces equation along the direction of the slope. $F_f$: force of friction, $m$: mass of ball
\begin{equation}
    ma = mg\sin\theta - F_{f}
\end{equation}
If you are at the CM, friction is the only force that provides a torque. $F_G$ does not affect the angular impact ($F_G$ and $r$ vectors are parallel, so cross product is 0). This gives 
\begin{align}
    \sum \tau &= I \alpha \nonumber \\
    I\alpha &= \vec{r} \times \vec{F_f}  \\
            &= rF_f \nonumber \\
    F_f     &= \frac{I\alpha}{r}
\end{align}
Finally, note if the ball rolls without slipping, we can say $a = \alpha r$. We have 3 unknowns and three equations. Solve for $a$, starting from Equation (1).
\begin{align}
    ma &= mg\sin\theta - F_{f} \nonumber \\
    ma &= mg\sin\theta - \frac{I\alpha}{r}\nonumber \\
    ma &= mg\sin\theta - \frac{I a}{r^2} \nonumber \\
    \intertext{sub $I = \frac{2}{5}mr^2$}
    ma &= mg\sin\theta - \frac{2 ma}{5} \nonumber \\
    a &= \frac{5}{7}g\sin\theta
\end{align}
Remember that $F_f = \mu\cdot mg\cos\theta$. Go back to Equation (1), sub in (4) and $F_f = \mu \cos\theta$. Solve for $\mu$.

\begin{align*}
    m \frac{5}{7} g \sin\theta &= mg\sin\theta - \mu mg\cos\theta \\
    \mu \cos\theta &= \frac{2}{7}\sin\theta \\
    \mu &= \frac{2}{7}\tan\theta
\end{align*}

\end{document}
