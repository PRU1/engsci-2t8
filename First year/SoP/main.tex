\documentclass[12pt]{article}
\usepackage[tmargin=0.75in, lmargin=0.75in, rmargin=0.75in, bmargin=0.75in]{geometry}
\usepackage{mypackagesv3}
\usepackage{setspace}
\usepackage{fancyvrb} % for "\Verb" macro
\usepackage{fancyhdr}

\setstretch{1.2}

\pagenumbering{gobble}
\begin{document}
\pagestyle{fancy}
\fancyhead[C]{\Large Pranav Upreti - Statement of Purpose}

My name is Pranav Upreti and I am a first-year Engineering Science hoping to major in Engineering Physics, with a passion for astrophysics. Unlike most, my first revelation in astrophysics was not identifying constellations or looking for planets in the night sky. My introduction to astrophysics was stumbling on \textit{A Brief History of Time} by Stephen Hawking, for the sole reason that the cover looked cool. 

Unfortunately, I was 8 years old when I picked up that book, and frankly, could not understand anything. But that was okay; it planted the seed of curiosity in me. I was curious and determined to one day understand what Hawking wrote. 

While I do not share the same understanding of astrophysics as Hawking, I am making steady progress towards it. I tend to have a deep-level curiosity for things, so I go beyond the simple Google and usually end up picking a few books to study. This inadvertently means I take long to learn a topic because I learn about several prerequisites, but I find it enjoyable. This curiosity is an asset to the research required at SURP.

I am most interested in:
\begin{enumerate}
    \item Project 3: Exploring New Frontiers in Galactic Archaeology with JWST
    \item Project 6: Development of instrumentation for radio astronomy applications
    \item Project 1: Turbulence in Low Rayleigh Number Red Giants
\end{enumerate}

Project 3 is my top pick since I have been learning machine-learning theory (primarily supervised learning), so it would be wonderful to apply it to parsing JWST data. Specifically, I have written neural networks using only math and \texttt{numpy}, in addition to experimenting with HuggingFace pre-trained LLMs and PyTorch CNNs. 

Additionally, I would love the opportunity to work with the professors leading the project, Prof. Li, Sandford, and Bovy. I recently came across Bovy's book on the astrophysics of galaxies, and hoping to learn from it soon.  On a more personal note, each professor has a career trajectory similar to what I am interested in, namely a physics-related undergrad and then a career research about galaxies.

I also applied to research projects developing radio equipment since I have experience in experimental design at a household scale. Namely, I designed, iterated, and built a thin-plate capacitor that I used to calculate the speed of light within 1\% error. I learned to be meticulous at every step of the build process, or else imperfections during construction propagate to the final result. No detail, even the specks of sawdust stuck in the dielectric, should go unmissed. These skills translate to the experimental design work required to build antenna and analog receiver hardware at higher levels of perfection in a research environment.

I am confident that I have the necessary skill set to contribute to any SURP project. I hope to gain a more intimate understanding of (i) what doing research at a high level and working with professors is like, and (ii) explore whether research is something I want to pursue as a career.

Thank you so much for reviewing my application, and I hope to hear back from you soon!


\end{document}
